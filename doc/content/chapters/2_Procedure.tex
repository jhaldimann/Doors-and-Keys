\chapter{Vorgehen}
In diesem Kapitel wollen wir Schritt für Schritt
aufzeigen, wie wir beim Lösen der einzelnene
Aufgaben vorgegenagen sind.

\section{Modellieren der Ausgangslage}
Bei der Modellierung der Ausgangslage, haben wir genau
die Situation, wie sie auf dem Aufgabenblatt steht,
in Prolog umgesetzt. Dabei haben wir für die 
Schlüssel und Tür Beziehung den Fact
\textbf{roomkey(X,Y)} verwendet. Wobei \textbf{X} der Raum ist in welchem sich
der Schlüssel \textbf{Y} befindet. So kann überprüft 
werden, ob eine Raum einen Schlüssel enthält.\\
\\

\noindent
Eine weitere wichtige Modellierung, welche zwingend notwendig
war, ist die Verschachtelung der Räume. Wir müssen wissen,
in welchem Raum sich welcher Unterraum befindet, damit wir an 
weitere Schlüssel kommen können. Deswegen haben wir den
Fact \textbf{roomcontainsroom(X,Y)} erstellt. Bei diesem ist \textbf{X} der äussere und \textbf{Y}
der innere Raum. Somit kann geprüft werden, ob ein Raum in einem 
anderen Raum vorhanden ist. Dies ist vorallem dann wichtig, wenn
wir wissen wollen wo sich ein bestimmter Raum befindet.\\
\\

\noindent
Zu guter Letzt haben wir noch den Raum festgelegt, in welchem sich 
der Schatz befindet, welchen man schlussendlich finden muss. Für dies
haben wir den Fact \textbf{roomcontainschest(X)}, wobei X der Raum ist
wo sich der Schatz befindet.

\section{Verwendete Rules}
In diesem Kapitel beschreiben wir alle von uns verwendeten Rules.
Dazu wird auch immer eine kleiner Codeausschnitt zu sehen sein.

\subsection{Union}
Bei der Union-Rule wird, wie der Name schon sagt, eine Vereinigung
aus zwei Listen gemacht. Diese verwenden wir anschliessend in der 
\textbf{cangetchestfrom} Rule.

\begin{verbatim}
    union([X|Y],Z,W) :- member(X,Z),  union(Y,Z,W).
    union([X|Y],Z,[X|W]) :- \+ member(X,Z), union(Y,Z,W).
    union([],Z,Z).
\end{verbatim}

\subsection{Member}
Mit dieser Rule überprüfen wir, ob sich ein Element bereits in 
unserer Liste befindet.

\begin{verbatim}
    member(X,[X|_]).
    member(X,[_|TAIL]) :- member(X,TAIL).
\end{verbatim}

\newpage

\subsection{ListAppend}
Um bei einer Liste ein Element hinzuzufügen haben wir die \textbf{list_append()} Rule geschrieben. Damit können wir alle Schlüssel, welche nötig sind um zum
Schatz zu kommen, in einer Liste anzeigen.

\begin{verbatim}
    list_append(A,T,T) :- member(A,T),!.
    list_append(A,T,X) :- append([A],T,X).
\end{verbatim}

\subsection{BubbleSort}
Um die Schlüssel in der Reihenfolge, in welcher sie geholt werden müssen, anzeigen zu können sortieren wir
die Liste der Schlüssel nach der Anzahl Schlüssel, welche benötigt werden um ihn zu erreichen.
Wir uns für den einfachsten Sortieralgorithmus 
entschieden. Den Bubblesort haben wie folgt umgesetzt:

\begin{verbatim}
    bubble_sort(List,Sorted):-b_sort(List,[],Sorted).
    b_sort([],Acc,Acc).
    b_sort([H|T],Acc,Sorted):-bubble(H,T,NT,Max),b_sort(NT,[Max|Acc],Sorted).
    
    bubble(X,[],[],X).
    bubble(X,[Y|T],[Y|NT],Max):-nth0(0,X,A), nth0(0,Y,B), A>B,bubble(X,T,NT,Max).
    bubble(X,[Y|T],[X|NT],Max):-nth0(0,X,A), nth0(0,Y,B), A=<B,bubble(Y,T,NT,Max).
\end{verbatim}

\subsection{CanGetKeyFrom}
Mit dieser Rule wollen wir feststellen, ob eine Schlüssel \textbf{K} von einem 
Raum \textbf{F} aus erreichbar ist. Wenn der Schlüssel erreicht werden kann, wird eine Liste \textbf{B}
mit allen notwendigen Schlüsseln zurückgegeben. \textbf{L} ist am Anfang eine leere Liste, welche mit 
\textbf{list_append()} befüllt wird.

\begin{verbatim}
    cangetkeyfrom(K,F,L,B):- (roomkey(F, K), list_append(K, L, B) ; 
        (K > 0, list_append(K, L, A), roomkey(X, K), cangetkeyfrom(X,F,A,C), 
        (roomcontainsroom(F,X), append([],C,B) ; 
        (roomcontainsroom(Y,X), cangetkeyfrom(Y,F,C,B))))).
\end{verbatim}

\subsection{CanReachRoomFrom}
Wie der Name bereits sagt, wollen wir herausfinden, ob ein Raum \textbf{R} von einem
anderen Raum \textbf{F} aus erreicht werden kann. Wenn dies möglich ist, wird die Funktionen
eine Liste \textbf{L} mit Schlüsseln zurückgeben. Ansonsten wird \textbf{false} als Resultat returniert.

\begin{verbatim}
    canreachroomfrom(R,F,L):- cangetkeyfrom(R,F,[],A), 
        (roomcontainsroom(F,R), append([],A,L) ; 
        (roomcontainsroom(Y,R), cangetkeyfrom(Y,F,[],L))).
\end{verbatim}

\subsection{CanGetChestFrom}
Mit dieser Rule stellen wir fest, ob wir den Schatz von einem bestimmten
Raum \textbf{F} aus finden können. Wenn Prolog eine Lösung gefunden hat, wird eine
Liste \textbf{L} mit benötigten Schlüsseln zurückgegeben. Diese Liste ist im Endeffekt
nichts anderes als eine Schritt für Schritt Anleitung, wie der Schatz erreicht werden 
kann. Am Ende der Liste wird zusätzlich noch ein \textbf{S} angefügt, welches den Schatz 
symbolisiert.

\begin{verbatim}
    cangetchestfrom(F,L):- roomcontainschest(X), cangetkeyfrom(X,F,[],A),
        canreachroomfrom(X,F,B), union(A,B,C), sortbykey(F,C,D), append(D,["S"],L).
\end{verbatim}

\newpage

\subsection{LengthKey}
Um unsere Reihenfolge der Schritte richtig zu bestimmen, haben wir eine Funktion geschrieben,
welche ein Tupel aus dem Schlüssel \textbf{K} und der Anzahl Schlüssel \textbf{N}, um den Schlüssel
zu erreichen, erstellt. So wissen wir genau wo in unserer Liste der entsprechende Schlüssel
abgelegt werden muss.

\begin{verbatim}
    lengthkey(K,O):- cangetkeyfrom(K,0,[],B), length(B,N), O=[N,K].
\end{verbatim}

\subsection{GetKeyOnly}
Diese Rule wird verwendet um den Schlüssel aus dem vorher erstellten Tupel herauszuholen.

\begin{verbatim}
    getkeyonly(LK,K):- nth0(1,LK,K).
\end{verbatim}

\subsection{SortByKey}
Zu guter Letzt, wollen wir noch die Liste der Tupels sortieren, welche in den vorherigen Rule
erstellt wurden. Dazu verwenden wir \textbf{maplist} und unseren Bubblesort Algorithmus.

\begin{verbatim}
    sortbykey(F,L,O):- maplist(lengthkey, L, A), 
        bubble_sort(A,S), maplist(getkeyonly, S, O).
\end{verbatim}