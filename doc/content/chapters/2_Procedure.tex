\chapter{Vorgehen}
In diesem Kapitel wollen wir Schritt für Schritt
angeben wie wir beim Lösen der einzelnene
Aufgaben vorgegenagen sind.

\section{Modelieren der Ausgangslage}
Bei der Modelierung der Ausgangslage, haben wir genau
die Situation, wie sie auf dem Aufgabenblatt steht,
in Prolog umgesetzt. Dabei haben wir für die 
Schlüssel und Tür Beziehung das Keyword
\textbf{doorkey()} verwendet. Dabei wird in der 
ersten Position in der Klammer der Raum und an der
zweiten Position der Schlüssel eingesetzt. So kann überprüft 
werden, ob eine Raum einen Schlüssel enthält.\\
\\

\noindent
Eine weitere wichtige Modelierung, welche zwingend notwendig
war, ist die Verschachtelung der Räume. Wir müssen wissen,
in welchem Raum sich welcher Unterraum befindet, damit wir an 
weitere Schlüssel kommen können. Deswegen haben wir die 
Funktion \textbf{roomcontainsroom} erstehlt, welche als erstes
Argument den übergeordneten Raum nimmt und an der zweiten Stelle
den Unterraum. Somit kann geprüft werden, ob ein Raum in einem 
anderen Raum vorhanden ist. Dies ist vorallem dann wichtig, wenn
wir wissen wollen wo sich ein bestimmter Raum befindet.\\
\\

\noindent
Zu guter Letzt haben wir noch den Raum festgelegt, in welchem sich 
der Schatz befindet, welchen man schlussendlich finden muss. Diese
Funktion wird auch nur verwendet um zu vergleichen, ob das Ziel 
erreicht wurde.